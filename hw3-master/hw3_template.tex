\documentclass{article}
% \usepackage[UTF8]{ctex}

% \usepackage[T1]{fontenc}
% \usepackage{CJKutf8}

\usepackage{bera}% optional: just to have a nice mono-spaced font
\usepackage{listings}
\usepackage{xcolor}
\usepackage{enumitem}

\colorlet{punct}{red!60!black}
\definecolor{background}{HTML}{EEEEEE}
\definecolor{delim}{RGB}{20,105,176}
\colorlet{numb}{magenta!60!black}

\lstdefinelanguage{json}{
    basicstyle=\normalfont\ttfamily,
    numbers=left,
    numberstyle=\scriptsize,
    stepnumber=1,
    numbersep=8pt,
    showstringspaces=false,
    breaklines=true,
    frame=lines,
    backgroundcolor=\color{background},
    literate=
     *{0}{{{\color{numb}0}}}{1}
      {1}{{{\color{numb}1}}}{1}
      {2}{{{\color{numb}2}}}{1}
      {3}{{{\color{numb}3}}}{1}
      {4}{{{\color{numb}4}}}{1}
      {5}{{{\color{numb}5}}}{1}
      {6}{{{\color{numb}6}}}{1}
      {7}{{{\color{numb}7}}}{1}
      {8}{{{\color{numb}8}}}{1}
      {9}{{{\color{numb}9}}}{1}
      {:}{{{\color{punct}{:}}}}{1}
      {,}{{{\color{punct}{,}}}}{1}
      {\{}{{{\color{delim}{\{}}}}{1}
      {\}}{{{\color{delim}{\}}}}}{1}
      {[}{{{\color{delim}{[}}}}{1}
      {]}{{{\color{delim}{]}}}}{1},
}

\usepackage{CJKutf8}
\usepackage[utf8]{inputenc}
\usepackage{amsmath}
\usepackage{amssymb}
\usepackage{clrscode3e}
\usepackage{mathtools}
\usepackage{nccmath}
\usepackage{hyperref}
\usepackage{geometry}
\usepackage{lipsum}
\usepackage{tocloft}
\geometry{
 a4paper,
 left=25mm,
 top=25mm,
}
\setcounter{section}{0} % start numbering at 1

\title{COMS 4111 Intro to DB HW3 - Part 2 \\
Normalization}
\author{Jing Qian (jq2282)}
% \subtitle{ASSIGNED: OCTOBER 18th, 2018}
\date{ASSIGNED: October 18th, 2018 \\
DUE: November 15th, 2018}

\begin{document}

\maketitle



\pagebreak

% Right!
\section*{Q2.1}
\paragraph{Write answers here.}
\begin{enumerate}[label=\alph*]
    \item All the non-trivial functional dependencies in the closure that have only one attribute on the right side are: $B \rightarrow C, C \rightarrow A, B \rightarrow A$.
    \item The minimal key of R is B.
\end{enumerate}

\newpage

\section*{Q2.2}
\paragraph{Write answers here.}
\begin{enumerate}[label=\alph*]
    \item All the non-trivial functional dependencies in the closure that have only one attribute on the right ride are: $AB \rightarrow D, BD \rightarrow C, CD \rightarrow A, AD \rightarrow B, AB \rightarrow C, BD \rightarrow A, AD \rightarrow C, CD \rightarrow B$.
    \item The minimal keys of S are AB, or BD, or AD, or CD.
\end{enumerate}

\newpage

\section*{Q2.3}
\paragraph{Write answers here.}
The keys in iowa are date, store, vendor\_no, itemno, invoice\_line\_no.

\section*{Q2.4}
\paragraph{Write answers here.}

For simplicity, we represent every attribute with one letter:\\
addree = A, bottle\_volumn\_ml = B, category = C, category\_name = D, city = E, county = F, county\_number = G, date = H, im\_desc = I, invoice\_line\_no = J, itemno = K, name = L, pack = M, sale\_bottles = N, sale\_dollars = O, sale\_gallons = P, sale\_liters = Q, state\_bottle\_cost = R, state\_bottle\_retail = S, store = T, store\_location\_address = U, store\_location\_city = V, store\_location\_zip = W, vendor\_name = X, vendor\_no= Y, vendor\_no = Z, store\_location = $\alpha$.

So the functional dependencies could be represented as: T -> AEFGLUVWZ$\alpha$, Y -> X, C -> D, K -> BCIRS, HKJTY -> MNOPQ.

And we could get the minimum cover: $F_{min} = $ \{T -> A, T -> E, T -> F, T -> G, T -> L, T -> U, T -> V, T -> W, T -> Z, T -> $\alpha$, Y -> X, C -> D, K -> B, K -> C, K -> I, K -> R, K -> S, HKJTY -> M, HKJTY -> N, HKJTY -> O, HKJTY -> P, HKJTY -> Q\}.

Then we do the composition as following: 

Applying 'Y -> X': 'YX', 'ABCDEFGHIJKLMNOPQRSTUVWYZ$\alpha$'.

Applying 'C -> D': 'YX', 'CD', 'ABCEFGHIJKLMNOPQRSTUVWYZ$\alpha$'.

Applying 'T -> AEFGLUVWZ$\alpha$': 'YX', 'CD', 'AEFGLTUVWZ$\alpha$', 'BCHIJKMNOPQRSTY'.

Applying 'K -> BCIRS': 'YX', 'CD', 'AEFGLTUVWZ$\alpha$', 'BCIKRS', 'HJKMNOPQTY'.

Check the $F_{min}$, all functional dependencies are covered and no redundancy.
So this is the 3NF decomposition for the table: 'YX', 'CD', 'AEFGLTUVWZ$\alpha$', 'BCIKRS', 'HJKMNOPQTY'.
Write in the name of attributes:

[vendor\_name, vendor\_no].

[category, category\_name].

[addree, city, county, county\_number, name, store, store\_location\_address, store\_location\_city, store\_location\_zip, zipcode, store\_location].

[bottle\_volumn\_ml, category, im\_desc, itemno, state\_bottle\_cost, state\_bottle\_retail].

[date, invoice\_line\_no, itemno, pack, sale\_bottles, sale\_dollars, sale\_gallons, sale\_liters, store, vendor\_no].


\section*{Q2.5}
\paragraph{Write answers here.}

Yes, my schema is free of redundancies and anomalies. 
All the sub-relations have been confirmed without redundancies.
Moreover, it is a 3NF decomposition and there are no lost joins (original relation could be recovered from the join of the sub-relations) or lost dependencies (checked with $F_{min}$).
So there is no anomalies.



\section*{Q2.6}
\paragraph{Write answers here.}
No, I can't. Because function dependencies are about relationship between attributes, hence they could not be used to constrain the value of one certain column.

\section*{Q2.7}
\paragraph{Write answers here.}
\begin{enumerate}[label=\alph*]
    \item There is only one vendor\_name value for itemno number '3326' in the iowa dataset.
    \item Yes, it should be. Because the same product should have the same vendor. The relationship between item and vendor should be one to many.
\end{enumerate}

\newpage
\end{document}
